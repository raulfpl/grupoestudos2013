\documentclass[11pt,a4paper]{report}

\usepackage[brazil]{babel}
\usepackage[T1]{fontenc}
\usepackage[latin1]{inputenc}
\usepackage{amsmath,amsfonts}
\usepackage{bussproofs}
\usepackage{amssymb}
\usepackage{tikz}

\usepackage{latexsym}
\usepackage{pgf}

\newcounter{conta}
 
\begin{document}
 
\noindent Grupo de Estudos de Programa\c{c}\~ao Funcional
 \hfill DECSI - UFOP \\
Professor: \parbox[t]{14cm}{Rodrigo Geraldo Ribeiro \\
                     e-mail: rodrigo@decsi.ufop.br}
 
\vspace*{3mm}
 O objetivo desta lista \'e implementar um conjunto de fun\c{c}\~oes Haskell que simula um interpretador de consultas
 em um sistema gerenciador de banco de dados. 
	           
	           Tabelas de um banco de dados s\~ao representadas pelo tipo de dados \texttt{Table}:
\begin{verbatim}
data Table = Table {
                tableName   :: String,   
                tableFields :: [Field],
                values :: [[String]]
           } deriving (Eq, Ord)
\end{verbatim}
Um valor do tipo Table \'e formado por tr\^es componentes, onde o primeiro componente \'e o nome da tabela, 
o segundo \'e a lista dos campos da tabela e o terceiro componente \'e a lista das linhas da tabela, sendo cada linha
uma lista em que cada elemento em uma posi\c{c}\~ao $i$ da lista corresponde \`a informa\c{c}\~ao da coluna $i$ . 
Todas as informa\c{c}\~oes contidas na tabela s\~ao do tipo String.

Cada campo de uma tabela \'e representado por um valor do tipo \texttt{Field}, que \'e formado pelo nome do campo e pelo
tipo de dados associado com este campo. Tipos de dados s\~ao representados pelo tipo \texttt{Type} e especificam os
 tipos de dados mais comuns presentes em bancos de dados relacionais. Os tipos \texttt{Field} e \texttt{Type} s\~ao
 apresentados abaixo:

\begin{verbatim}
data Field = Field  {
                fieldName :: String, 
                fieldType :: Type    
           }           
           deriving (Eq, Ord)

data Type = TyInt
          | TyDouble                         
          | TyBool                           
          | TyVarChar (Maybe Int)             
          | TyDate                           
          | TyCurrency                       
          deriving (Eq, Ord)
\end{verbatim}

Para exemplificar a representa\c{c}\~ao a ser utilizada neste trabalho, considere o trecho de c\'odigo a seguir, onde \'e
mostrada uma tabela de exemplo:

\begin{verbatim}
client :: Table
client = Table "Cliente" fieldsCliente dataCliente

fieldsCliente :: [Field]
fieldsCliente = [Field "id" TyInt, 
                 Field "nome" (TyVarChar (Just 15)), 
                 Field "cpf" (TyVarChar (Just 11))]

dataCliente :: [[String]]
dataCliente = [["1", "Jose da Silva", "23333245678"], 
               ["2", "Joaquim Souza", "09863737213"], 
               ["3", "Roberto Martins", "45627819081"]]
\end{verbatim}

Observe que os valores \texttt{fieldsCliente} e \texttt{dataCliente} s\~ao utilizados apenas para criar a lista de campos
e dos valores associados com a tabela criada.
\begin{flushleft}
\textbf{Exerc\'icio 1}: Um esquema de uma tabela de um banco de dados relacional \'e um produto dos tipos de cada uma 
                        das tabelas de um banco de dados. 
                       Considere o seguinte tipo de dados que representa um esquema de uma tabela de um banco de dados
                       relacional:
\end{flushleft}
	\begin{verbatim}
data Schema = Type :*: Schema 
            | Nil 
            deriving (Eq, Ord)
	\end{verbatim}

Onde o construtor de dados \texttt{:*:} representa o produto do tipo de um campo de uma tabela e o esquema que
representa o restante da tabela. Como exemplo, o esquema que representa a tabela cliente mostrada anteriormente \'e:

\begin{verbatim}
esquemaCliente :: Schema
esquemaCliente = TyInt                   :*: 
                 (TyVarChar (Just 15))   :*:
                 ((TyVarChar (Just 11))) :*: Nil
\end{verbatim} 	

O construtor de dados \texttt{Nil} \'e utilizado apenas como um marcador de fim do produto do esquema de uma tabela.

Neste primeiro exerc\'icio voc\^e dever\'a desenvolver:

\textbf{Exerc\'icio 1-a)}: O tipo (esquema) de uma tabela \'e representado pelo produto cartesiano de cada uma das 
colunas que comp\~oe a tabela. Valores do tipo \texttt{Type} podem ser impressos (convertidos em Strings) de acordo com
 a seguinte tabela de equival\^encia:
 
 
\begin{tabular}{|l|l|}
 	\hline
 	   Valor do tipo \texttt{Type} & String correspondente\\ \hline
 	   \texttt{TyInt}              & \texttt{Int}\\
 	   \texttt{TyDouble}           & \texttt{Double}\\
 	   \texttt{TyBool}             & \texttt{Bool}\\
 	   \texttt{TyVarChar (Just n)} & \texttt{VarChar[n]}\\
 	   \texttt{TyVarChar Nothing}  & \texttt{VarChar}\\
 	   \texttt{TyDate}             & \texttt{Date}\\
 	   \texttt{TyCurrency}         & \texttt{Currency}\\
 	\hline
\end{tabular}
\begin{verbatim}

\end{verbatim} 

Para a impress\~ao de esquemas, basta considerar que o construtor de dados \texttt{:*:} \'e o equivalente ao produto
cartesiano \texttt{X} entre tipos de cada coluna de uma tabela. Como exemplo, o esquema da tabela cliente, apresentado 
anteriormente, pode ser impresso como: \texttt{Int X VarChar[15] X VarChar[11]}. Observe que o construtor de dados 
\texttt{Nil} \'e substitu\'ido pela string vazia. Tendo como base as informa\c{c}\~oes apresentadas, fa\c{c}a o que se
pede:

\textbf{Exerc\'icio 1-a-i)} Desenvolva uma inst\^ancia de \texttt{Show} para o tipo \texttt{Type} que reflita a 
convers\~ao de valores deste tipo em strings de acordo com a tabela apresentada anteriormente.

\textbf{Exerc\'icio 1-a-ii)} Desenvolva uma inst\^ancia para \texttt{Show} para o tipo \texttt{Schema} que permita
imprimir esquemas de tabelas de maneira similar ao exemplo apresentado anteriormente para a tabela cliente.


\textbf{Exerc\'icio 1-b)}: Uma fun\c{c}\~ao \texttt{schema :: Table -> Schema} que dada uma tabela, representada pelo 
tipo de dados \textbf{Table}, retorne o esquema correspondente a esta. 

\textbf{Exerc\'icio 2}: Defina uma inst\^ancia de \texttt{Show} para o tipo \texttt{Table}, de maneira que a tabela 
cliente apresentada anteriormente seja impressa da seguinte maneira:
\begin{verbatim}
Cliente:
--------------------------------
id nome            cpf         
--------------------------------
1  Jose da Silva   23333245678 
2  Joaquim Souza   09863737213 
3  Roberto Martins 45627819081 
\end{verbatim}

\textbf{Exerc\'icio 3}: Defina uma fun\c{c}\~ao \texttt{count :: Table -> Int} que retorne o n\'umero de registros
presentes em uma tabela. Se aplicarmos a fun\c{c}\~ao \texttt{count} \`a tabela de clientes, o resultado dever\'a ser
\texttt{3}.

\textbf{Exerc\'icio 4}: Defina uma fun\c{c}\~ao:
\begin{center}
 \texttt{project :: Table -> [String] -> Either String Table}
\end{center}

que receba como argumento uma tabela e uma lista de nomes de campos desta tabela e retorne como resultado uma nova 
tabela que contenha somente as informa\c{c}\~oes presentes nas colunas especificadas pelo segundo argumento. Como 
exemplo, se executarmos \texttt{project client ["id","nome"]} o resultado ser\'a:
\begin{verbatim}
Right Cliente-id-nome:
-------------------
id nome            
-------------------
1  Jose da Silva   
2  Joaquim Souza   
3  Roberto Martins
\end{verbatim}

Observe que o nome da tabela resultante \'e formado pela concatena\c{c}\~ao do nome da tabela com o nome dos campos 
que foram fornecidos como argumento para a fun\c{c}\~ao \texttt{project}. O resultado desta fun\c{c}\~ao \'e expresso
pelo tipo \texttt{Either String Table}. O construtor de tipos \texttt{Either} \'e normalmente utilizado para representar
fun\c{c}\~oes que podem resultar em erros. No caso da fun\c{c}\~ao \texttt{project}, ela deve retornar uma mensagem de
erro, sempre que seja passado como par\^ametro um nome de campo que n\~ao esteja presente na tabela na qual deseja-se
realizar a proje\c{c}\~ao. Isto \'e, se executarmos \texttt{project client ["id", "teste"]} dever\'a ser retornado o
seguinte resultado:
\begin{verbatim}
Left "The fields:\nteste\nisnt defined in table: Cliente"
\end{verbatim}
onde \texttt{Left} \'e o construtor de dados do tipo \texttt{Either} que \'e utilizado para representar erros.

\textbf{Exerc\'icio 5}: Defina a fun\c{c}\~ao:
\begin{center}
\texttt{restrict :: Table -> Condition -> Either String Table}
\end{center}
Que seleciona todos os registros de uma tabela que atendem a uma determinada condi\c{c}\~ao. Condi\c{c}\~oes s\~ao 
representadas pelo tipo \texttt{Condition}, apresentado abaixo:
\begin{verbatim}
data Condition = Condition {
                   field :: String,
                   condition :: String -> Bool
               }
\end{verbatim}
onde \texttt{field} representa o nome do campo que ser\'a utilizado no processo de sele\c{c}\~ao de registros e 
\texttt{condition} \'e uma fun\c{c}\~ao que verifica se um valor deste campo atende ou n\~ao a condi\c{c}\~ao
 especificada. Como exemplo, considere o seguinte valor do tipo \texttt{Condition}, que representa a condi\c{c}\~ao:
 \textit{Selecione todos os nomes que come\c{c}am com a letra 'R'}:
\begin{verbatim}
cond :: Condition
cond = Condition "nome" (\s -> head s == 'R')
\end{verbatim}

Ao executarmos \texttt{restrict client cond} deve ser impresso o seguinte valor:

\begin{verbatim}
Right Cliente:
--------------------------------
id nome            cpf         
--------------------------------
3  Roberto Martins 45627819081 
\end{verbatim}

\textbf{Exerc\'icio 6}: A fun\c{c}\~ao a seguir possibilita combinar duas tabelas. Duas tabelas apenas podem ser
combinadas se elas possuem um campo em comum, isto \'e, uma coluna com o mesmo nome e tipo.

A tabela resultante da combina\c{c}\~ao de duas tabelas deve possuir todas as colunas da primeira tabela, seguidas
pelas colunas da segunda tabela que n\~ao ocorrem na primeira tabela. Portanto, o n\'umero de colunas na tabela
resultante \'e menor do que a soma do numero de colunas de cada uma das tabelas combinadas.
Cada registro na nova tabela \'e formado pela combina\c{c}\~ao dos registros correspondentes das duas tabelas 
combinadas, ou seja, por registros que t\^em o mesmo valor na coluna comum \`as duas tabelas. O nome da nova tabela
\'e formado pela concatena\c{c}\~ao dos nomes das duas tabelas que s\~ao combinadas.
Para isso, defina a fun\c{c}\~ao:
\begin{center}
\texttt{join :: Table -> Table -> Either String Table}
\end{center}
que realize a jun\c{c}\~ao de duas tabelas como explicado acima.

Para exemplificar, considere a seguinte tabela que representa endere\c{c}os de clientes:

\begin{verbatim}
Endereco:
----------------------------------------------------
cpf         rua    bairro cidade          estado 
----------------------------------------------------
23333245678 rua 2  alfa   jurema do sul   SC     
09863737213 rua 89 beta   jurema do norte SC     
45627819081 rua 10 gama   jurema do leste SC     
\end{verbatim}
Observe que ela possui um campo em comum com a tabela cliente (o campo \texttt{cpf}). O resultado de exercutarmos 
\texttt{join client address}, onde \texttt{address} \'e o valor que representa a tabela de endere\c{c}os acima, deve ser:

\begin{verbatim}
Right Cliente-Endereco:
-------------------------------------------------------------------------
id nome            cpf         rua    bairro cidade          estado 
-------------------------------------------------------------------------
1  Jose da Silva   23333245678 rua 2  alfa   jurema do sul   SC     
2  Joaquim Souza   09863737213 rua 89 beta   jurema do norte SC     
3  Roberto Martins 45627819081 rua 10 gama   jurema do leste SC  
\end{verbatim}
\'E importante ressaltar que a ordem das par\^ametros fornecidos para a opera\c{c}\~ao de jun\c{c}\~ao altera o 
resultado. O resultado de executarmos \texttt{join address client} \'e:

\begin{verbatim}
Right Endereco-Cliente:
-------------------------------------------------------------------------
cpf         rua    bairro cidade          estado id nome            
-------------------------------------------------------------------------
23333245678 rua 2  alfa   jurema do sul   SC     1  Jose da Silva   
09863737213 rua 89 beta   jurema do norte SC     2  Joaquim Souza   
45627819081 rua 10 gama   jurema do leste SC     3  Roberto Martins 
\end{verbatim}
A opera\c{c}\~ao de jun\c{c}\~ao pode ser decomposta em dois passos:
\begin{enumerate}
	\item Calcular o produto cartesiano das duas tabelas. Para isso, \'e criada uma nova tabela que possuir\'a como
	      campos todos os campos de ambas as tabelas envolvidas no produto. Os registros s\~ao obtidos realizando
	      todas as combina\c{c}\~oes de ambas as tabelas. Isto \'e feito pegando cada registro da primeira tabela e 
	      concatenando-o com cada registro correspondente na segunda tabela.
	\item Selecionar todos os registros onde o campo em comum de duas tabelas possui o mesmo valor.
\end{enumerate}
Para implementar a fun\c{c}\~ao \texttt{join}, basta:
\begin{enumerate}
	\item Implementar uma fun\c{c}\~ao \texttt{cartProd :: Table -> Table -> Either String Table}, que calcula o produto
	      cartesiano de duas tabelas conforme descrito acima. Esta fun\c{c}\~ao deve retornar uma mensagem de erro,
	      caso as duas tabelas fornecidas como par\^ametro n\~ao possu\'irem um campo em comum.
	\item Implementar uma fun\c{c}\~ao \texttt{restrictProd :: Table -> Either String Table}, que seleciona os registros
	      adequados do produto cartesiano, conforme descrito acima.
	\item Implementar uma inst\^ancia de \texttt{Monad (Either a)} para permitir que opera\c{c}\~oes sobre o tipo de 
	      dados \texttt{Either a b} possam ser compostas.
\end{enumerate}
\textbf{Exerc\'icio 7}: Verifica\c{c}\~ao de tipos para tabelas

Um problema comum a todas as bibliotecas para manipula\c{c}\~ao de bancos de dados em quaisquer linguagens de 
programa\c{c}\~ao \'e que estas n\~ao garantem que valores inseridos / modificados em um banco de dados possuam
tipos adequados. Se alguma verifica\c{c}\~ao ocorre, esta \'e feita pelo banco de dados durante a execu\c{c}\~ao do
programa em quest\~ao. At\'e o presente ponto a modelagem de tabelas utilizada nos exerc\'icios anteriores n\~ao garante
 que os registros presentes na tabela possuam tipo que esteja de acordo com o esquema especificado por uma tabela. O 
 objetivo deste exerc\'icio \'e o desenvolvimento de uma fun\c{c}\~ao que verifique se todos os registros associados a
 uma tabela de um banco de dados est\~ao de acordo com o esquema da tabela em quest\~ao. Para isso considere o seguinte
 tipo que representa uma tabela cujos registros s\~ao bem tipados:
 \begin{verbatim}
data TypedTable = TypedTable {
                    tyTableName   :: String,
                    tyTableSchema :: Schema,
                    typedValues   :: [[Value]]   
                } deriving (Eq, Ord)
 \end{verbatim}
 
Observe que em uma tabela bem tipada, valores n\~ao s\~ao representados por strings mas sim por valores do tipo 
\texttt{Value}, que \'e apresentado abaixo:

\begin{verbatim}
data Value = IntVal Int
           | DoubleVal Double
           | BoolVal Bool
           | VarCharVal String
           | DateVal Date
           | CurrencyVal Currency 
           deriving (Eq, Ord)
\end{verbatim}

O tipo \texttt{Value} possui um construtor de dados para cada um dos poss\'iveis tipos que cada campo de uma tabela 
pode possuir. Para isso defina as seguintes fun\c{c}\~oes:
\begin{enumerate}
	\item \texttt{parseValue :: String -> Value} : que recebe uma String e retorna como resultado o valor correspondente.
	\item \texttt{parseRecord :: [String] -> [Value]}: que recebe uma lista de Strings correspondentes \`as linhas da
	      tabela e retorna uma lista de valores bem tipados correspondentes.
	\item \texttt{tcRecord :: [Value] -> Schema -> Either String [Value]}: que recebe uma lista de valores e um esquema
	      de uma tabela e retorna como resultado uma mensagem de erro, no caso de algum registro n\~ao estar de acordo
	      com o esquema da tabela ou os pr\'oprios valores no caso destes serem bem tipados.
	\item \texttt{tcTable :: Table -> Either String TypedTable}: que recebe uma tabela n\~ao tipada e retorna como 
	      resultado uma mensagem de erro ou uma tabela bem tipada.
\end{enumerate}
\textbf{Observa\c{c}\~oes}: Procure usar um bom estilo de programa\c{c}\~ao ao definir suas fun\c{c}\~oes e use
 coment\'arios claros para indicar a funcionalidade de cada defini\c{c}\~ao. \'E recomend\'avel o uso de todo o 
 conte\'udo visto ao longo do curso (fun\c{c}\~oes de ordem superior, list comprehensions). Al\'em disso, \'e 
 recomend\'avel que voc\^e divida seu programa em pequenas fun\c{c}\~oes para que voc\^e possa test\'a-las 
 separadamente. 
 
O m\'odulo que cont\'em o programa deve ter um coment\'ario inicial contendo os nomes dos componentes do grupo. 
O trabalho deve ser entregue at\'e a data especificada no site, devendo ser enviado por e-mail, para o
endere\c{c}o rodrigogribeiro at gmail.com, com o subject LP-Trabalho2 (isso ser\'a avaliado!).


\end{document}
