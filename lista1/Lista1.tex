\documentclass[a4paper,11pt]{exam}
\usepackage[utf8]{inputenc}
\usepackage[brazil]{babel}
\usepackage{amsmath}
\usepackage{amsfonts}
\usepackage{listings}
\usepackage{graphicx}
\usepackage{wasysym}
\usepackage{amssymb}
\usepackage{color}
\usepackage[usenames,dvipsnames,svgnames,table]{xcolor}


\begin{document}
\topmargin=0pt
\renewcommand{\solutiontitle}{\noident \textbf{Solução:} \par }

\section{Programação Funcional em Haskell --- Lista 1}
\begin{questions}

  \question Defina a função \texttt{double :: Int ->  Int}, que dado um inteiro, 
  retorna 2 vezes o valor do argumento.
  
  \question Defina a função  \texttt{double2 :: Int -> Int}, usando \texttt{double}, que
  retorna 4 vezes o valor do argumento.
  
  \question Defina a função \texttt{sel :: Bool -> Int -> Int -> Int}, que se comporta do
   seguinte modo: Se o valor do primeiro argumento for \texttt{false}, a função deve retornar o segundo 
   argumento, se o valor do primeiro argumento for \texttt{true}, a função deve retorna o terceiro argumento.\\

  \question Defina a função \texttt{max2 :: Int -> Int -> Int} que retorna o maior entre o dois parâmetros: 
   \begin{itemize}
      \item Usando a função \texttt{sel} do exercício 3.
      \item Sem usar a função \texttt{sel} do exercício 3. 
   \end{itemize}
   
  \question Defina \texttt{max3 :: Int -> Int -> Int -> Int} que retorna o maior entre 3 parâmetros, usando
  \texttt{max2}.

  \question Defina a função \texttt{eq2 :: Int -> Int -> Bool} que retorna \texttt{true} se ambos os argumentos
   são iguais e \texttt{false} caso contrário. \textbf{OBS:} Lembre-se que Haskell \textbf{não} unifica variáveis!

  \question Defina a função  \texttt{diferent3 :: Int -> Int -> Int -> Bool} que retorna \texttt{true} se os 3 argumentos
   forem diferentes, e \texttt{false} caso contrário. 
    
 
  \question  O algoritmo de Euclides pode ser usado para achar o maior divisor comum entre dois números.
  Matematicamente o algoritmo de Euclides pode ser definido do seguinte modo: 
   \begin{center} 
      \begin{tabular}{lcl}
         $gcd(a,0)$ & = & $a$ \\
         $gcd(a,b)$ & = & $gcd(b,a \;mod\; b)$\\
      \end{tabular}
   \end{center}
   Onde $mod$ é a operação que retorna o resto da divisão.\\ 
   \textbf{OBS:} Em Hsakell,  a função \texttt{mod} retorna o resto da divisão do primeiro
   argumento pelo segundo.
   \begin{parts}
      \part Implemente o algoritmo de Euclides em Haskell.
      \part Implemente a função que computa o mínimo múltiplo comum entre dois números, definido como:
\[
      \begin{array}{lcl}
         lcm(a,b) & = &\displaystyle \frac{a\,\cdot\,b}{gcd(a,b)}
      \end{array}
\]  
   \end{parts}
   
  
  \question  A função 91 de McCarthy é definida do seguinte modo:
  \[ 
    m(x)\; = \; \left\{
    \begin{array}{l l}
        n-10 \quad & \text{se $n$ $>$ 100}\\
        m(m(n+11)) \quad & \text{se $n$ $\leq$ 100 }\\
    \end{array} \right.
  \]
  Defina esta função em Haskell. Qual valor esta função retorna para valores de entrada positivos menores que 101 ?
  
  
  \question  Uma fração pode ser representada por pares de valores inteiros, onde o primeiro elemento do par representa
  o numerador e o segundo elemento do par representa o denominador. Por exemplo as seguintes frações
  \begin{center}
     \[\displaystyle \frac{3}{5},\quad  \frac{2}{8} \quad e \quad \frac{3}{4} \]
  \end{center}
  poderiam se expressas em Haskell como :
  \begin{center}
     (3,5), (2,8) e (3,4)
  \end{center}
  Considerado esta representação defina: 
  \begin{parts}
     \part A função \texttt{multf :: (Int,Int) -> (Int,Int) -> (Int,Int)} que multiplica duas frações.\\
      Ex.:\\ 
      GHCi$>$  multf (3,5) (2,8)\\
      (6,40)
     \part A função \texttt{somaf :: (Int,Int) -> (Int,Int) -> (Int,Int)} que soma duas frações.\\
      Ex.:\\ 
      GHCi$>$ somaf (1,2) (1,4)\\
       (6,8)\\
      \part A função \texttt{subf :: (Int,Int) -> (Int,Int) -> (Int,Int)} que subtrai duas frações.\\
      \part A função \texttt{divf :: (Int,Int) -> (Int,Int) -> (Int,Int)} que divide duas frações.\\
      \part A função \texttt{toReal :: (Int,Int) -> Float} que converte a fração em um número real.\\
  \end{parts}
  
   \question Defina a função \texttt{intercala :: [Int] -> [Int] -> [Int]} que intercala duas listas de inteiros, 
    do seguinte modo :\\
    GHCi$>$\texttt{intercala [1,2,3] [7,8]}\\
    \texttt{[1,7,2,8,3]}
   
     
	  \item Defina a fun\c{c}\~ao \texttt{quads :: Int -> Int -> [Int]} que retorna a lista formada pelos quadrados
	        dos n\'umeros presentes no intervalo n\'umerico definido pelo primeiro e segundo argumento. Isto \'e
	        \texttt{quads 1 5 $\rhd$ [1,4,9,16,25]}
	  \item Defina a fun\c{c}\~ao \texttt{sumQuads :: Int -> Int -> Int} que retorna a soma dos quadrados dos n\'umeros
	        definidos pelo intervalo do primeiro e segundo par\^ametro.  
	  \item Defina uma fun\c{c}\~ao \texttt{divisors :: Int -> [Int]} de maneira que \texttt{divisors n} retorne todos os
	        divisores do n\'umero \texttt{n}.
	  \item Um n\'umero \'e dito ser perfeito se ele \'e igual a soma de seus divisores. Defina uma fun\c{c}\~ao 
	        \texttt{perfect :: Int -> Bool} que retorna verdadeiro se o n\'umero passado como argumento \'e um 
	        n\'umero perfeito.
	  \item Um n\'umero \'e dito ser primo se ele \' divis\'ivel por \texttt{1} e por ele pr\'oprio. 
	        Defina uma fun\c{c}\~ao \texttt{prime :: Int -> Bool} que
                retorne verdadeiro se um n\'umero \'e primo.
          \item Defina a função \texttt{primes :: Int -> [Int]} que retorna
            todos os primos no intervalo $[1,n]$, onde $n$ \'e o par\^ametro
            para a função \texttt{primes}.
	  \item Defina uma fun\c{c}\~ao \texttt{media :: [Int] -> Float} que calcule a m\'edia dos valores de uma lista.
	  \item Uma lista \'e um pal\'indromo se esta for igual ao seu
            inverso. Defina uma fun\c{c}\~ao \texttt{palindrome}, tal que \texttt{palindrome xs} retorne verdadeiro se a lista
	        \texttt{xs} \'e um pal\'indromo. Qual o tipo desta fun\c{c}\~ao?
	  \item Defina uma fun\c{c}\~ao \texttt{toPalindrome} que receba uma lista e a transforme em um pal\'indromo caso
	        esta n\~ao seja um. Exemplos \texttt{toPalindrome [1,2,3] $\rhd$ [1,2,3,3,2,1]} e 
	        \texttt{toPalindrome [1,2,1] $\rhd$ [1,2,1]}. Qual o tipo
                desta fun\c{c}\~ao? 
           \item[\ ]
           \item[\ ] Essa lista foi elaborada em conjunto com o prof. Elton
             M\'aximo Cardoso.

\end{questions}


\end{document}
